\chapter{Implementazione}
\label{chp:implementazione}
In questo capitolo viene definito come gli algoritmi definiti nel capitolo \ref{chp:stato-arte} 
sono stati implementati all'interno di un framework in \texttt{C++}.
Il framework ha come scopo quello di permettere allo sviluppatore/ricercatore di implementare dei nuovi algoritmi
di sketching, inizialmente per il problema del \textit{count distinct}, poi eventualmente generalizzato e ottenere
in automatico dei test robusti sulla valutazione degli algoritmi, potendoli comparare empiricamente.

\section{Architettura del framework}
\subsection{Panoramica dei componenti}
Algoritmi in C++
Pipeline di sperimentazione: generazione/preprocessing → partizionamento → input al runner C++ → loop algoritmi → metriche → output CSV.
Scrittura dei dataset tramite script python ed eventuale partizionamento pre-fatto

\subsection{Flusso end-to-end di un esperimento}
Il framework scritto in c++ prende direttamente il file di input con le sue partizioni ed esegue tutti gli algoritmi
identificati in loop prendendo le eventuali misurazioni del caso.
\subsection{Scelte progettuali}


\section{Dataset}
Qua verra' discussa come i dati vengono generati e gestiti.
\subsection{Origine dei dati, formati e parsing}
\subsection{Generazione e partizionamento}

\section{Componenti comuni}
Qua verra' descritti gli elementi comuni di ogni algoritmo implementato
\subsection{Interfaccia Algorithm}
Operazioni supportate all'interno di un algoritmo
\subsection{Hashing}
Algoritmo / algoritmi di hashing supportati
\subsection{Misure di memoria e serializzazione}

\section{Algoritmi}
Qua verra' descritto come e' stato implementato ogni singolo algoritmo
Parametri esposti
Stato interno e layout memoria (footprint)
Update (costo e operazioni chiave)
Query (formula usata, correzioni applicate)
Merge (se supportato) + prerequisiti (seed/parametri)
Note pratiche (range, saturazioni, edge-case)

\subsection{Naive Counting}
\subsection{Probabilistic Counting}
Parametri e come e' stato implementato e core del codice importante
\subsection{LogLog}
Parametri e come e' stato implementato e core del codice importante
\subsection{HyperLogLog}
Parametri e come e' stato implementato e core del codice importante
\subsection{HyperLogLog++}
Parametri e come e' stato implementato e core del codice importante

\subsection{CountMin Sketch}
Parametri e come e' stato implementato e core del codice importante

\subsection{Bloom Filter}
Parametri e come e' stato implementato e core del codice importante

\section{Framework di benchmark}
Descrizione del framework di benchmark, come funziona, quale e' il flusso
\subsection{CLI e configurazione parametri}
Operazioni da CLI che si possono fare all'interno del framework
\subsection{Protocollo sperimentale}
\subsection{Metriche misurate}
Come sono state implementate le statistiche che sono state descritte all'interno del capitolo \ref{chp:stato-arte}
\subsection{Formato output e organizzazione dei risultati}
Come vengono visualizzati i risultati per ogni run degli algoritmi, dove vengono salvati?
Come vengono visti?
