\chapter{Implementazione}
\label{chp:implementazione}
In questo capitolo viene definito come gli algoritmi definiti nel capitolo \ref{chp:stato-arte} 
sono stati implementati all'interno di un framework in \texttt{C++}.
Il framework ha come scopo quello di permettere allo sviluppatore/ricercatore di implementare dei nuovi algoritmi
di sketching, inizialmente per il problema del \textit{count distinct}, poi eventualmente generalizzato e ottenere
in automatico dei test robusti sulla valutazione degli algoritmi, potendoli comparare empiricamente.

\section{Architettura Software}
Algoritmi in C++

Dataset in Python

Scrittura dei dataset tramite script python ed eventuale partizionamento pre-fatto

Il framework scritto in c++ prende direttamente il file di input con le sue partizioni ed esegue tutti gli algoritmi
identificati in loop prendendo le eventuali misurazioni del caso.

Volendo idea: sarebbe da prendere i dati in una stream continua e fare misurazione real-time di quello che
accade, ovviamente non si puo' sapere il valore esatto in questo caso.

Pero' si potrebbe invece fare che la stream esiste gia' pero' vengono dati un valore alla volta, in modo tale
da avere effettivamente i valori.

Pensare anche all'implementazione di un sistema reale di streaming di dati come Apache Kafka.

\section{Dataset}
\subsection{Generazione dataset}
\subsection{Generazione partizioni}

\section{Interfacce e componenti comuni}
\subsection{Interfaccia Algorithm}
\subsection{Hashing}

\section{Algoritmi}
\subsection{Naive Counting}
\subsection{Probabilistic Counting}
\subsection{LogLog}
\subsection{HyperLogLog}
\subsection{HyperLogLog++}

\subsection{CountMin Sketch}

\subsection{Bloom Filter}

\section{Framework di benchmark}
\subsection{Statistiche misurate}
