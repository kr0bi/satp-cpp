%!TEX root = ../dissertation.tex

\chapter{Risultati sperimentali}
\label{chp:risultati}

Questo capitolo presenta i risultati sperimentali finali ottenuti con il
framework descritto nel Capitolo~\ref{chp:implementazione}.
L'analisi è limitata alle modalità \emph{streaming} e \emph{merge}, mentre i
file \emph{oneshot} non vengonon utilizzati poiché, a differenza, della modalità streaming viene preso un solo risultato finale per esecuzione.
Le metriche considerate sono quelle formalizzate nel
Capitolo~\ref{chp:background}.

\section{Perimetro sperimentale}

I risultati sono estratti dai CSV presenti in \codepath{results/} con la
seguente copertura:
\begin{itemize}
    \item 72 file \codepath{results\_streaming.csv} (288000 righe);
    \item 72 file \codepath{results\_merge.csv} (36000 righe);
    \item \(n=10^7\) elementi per partizione;
    \item \(F_0 \in \{10^5,10^6,5\cdot10^6,10^7\}\);
    \item seed \(\{42,137357,10032018,21041998,29042026\}\);
    \item 50 run per configurazione.
\end{itemize}

La produzione delle figure e delle tabelle è eseguita con
\codepath{notebooks/12\_ch5\_best\_config\_analysis.ipynb},
che salva i grafici in \codepath{thesis/figures/results/} e le tabelle di
supporto in \codepath{notes/}.

\section{Selezione delle configurazioni migliori}

Il confronto principale è svolto fissando, per ciascun algoritmo, il parametro
che minimizza il \emph{mean relative error} medio al checkpoint finale
(aggregazione su seed e valori di \(F_0\)).
Le configurazioni selezionate sono:
\begin{itemize}
    \item HyperLogLog++: \(k=18\);
    \item HyperLogLog: \(k=16, L=32\);
    \item LogLog: \(k=15, L=32\);
    \item Probabilistic Counting: \(L=23\).
\end{itemize}

\begin{table}[htbp]
    \centering
    \small
    \setlength{\tabcolsep}{6pt}
    \renewcommand{\arraystretch}{1.12}
    \begin{tabular}{|l|c|r|}
        \hline
        Algoritmo & Parametro migliore & MRE medio endpoint \\
        \hline
        HyperLogLog & \(k=16\) & 0.002736 \\
        \hline
        HyperLogLog++ & \(k=18\) & 0.001370 \\
        \hline
        LogLog & \(k=15\) & 0.005615 \\
        \hline
        Probabilistic Counting & \(L=23\) & 0.524951 \\
        \hline
    \end{tabular}
    \caption{Configurazione migliore per algoritmo nel dominio esplorato.}
    \label{tab:best-config-selection}
\end{table}

\section{Risultati streaming con best-config}

\subsection{Stima media nel tempo e transitorio iniziale}

Le Figure~\ref{fig:best-stream-est-linear} e
\ref{fig:best-stream-est-loglog} mostrano l'andamento di
\(\hat{F}_0(t)\) rispetto a \(F_0(t)\) con seed 21041998.
La scala log-log è usata per rendere visibile il comportamento sui primi
checkpoint, come richiesto in fase di revisione.

\begin{figure}[htbp]
    \centering
    \includegraphics[width=0.96\textwidth]{figures/results/best_stream_estimate_vs_truth_linear_seed21041998.png}
    \caption{Best-config: confronto tra stima media \(\hat{F}_0(t)\) e valore reale \(F_0(t)\) in scala lineare (seed 21041998).}
    \label{fig:best-stream-est-linear}
\end{figure}

\begin{figure}[htbp]
    \centering
    \includegraphics[width=0.96\textwidth]{figures/results/best_stream_estimate_vs_truth_loglog_seed21041998.png}
    \caption{Stesso confronto della Figura~\ref{fig:best-stream-est-linear} in scala log-log.}
    \label{fig:best-stream-est-loglog}
\end{figure}

In questa configurazione HyperLogLog++ e HyperLogLog seguono da vicino la curva
reale su tutti i livelli di cardinalità testati.
LogLog rimane vicino ai due algoritmi HLL per \(F_0\) medio-alti, ma mantiene
uno scarto più evidente nel caso \(F_0=10^5\).
Probabilistic Counting resta l'algoritmo meno stabile nella fase transitoria,
con deviazioni significative già a metà stream.

\subsection{Varianza nel tempo}

La Figura~\ref{fig:best-stream-var-loglog} confronta la varianza della stima in
scala log-log.

\begin{figure}[htbp]
    \centering
    \includegraphics[width=0.96\textwidth]{figures/results/best_stream_variance_loglog_seed21041998.png}
    \caption{Best-config: varianza della stima in modalità streaming (seed 21041998, scala log-log).}
    \label{fig:best-stream-var-loglog}
\end{figure}

La dispersione di HyperLogLog++ è inferiore a HyperLogLog e LogLog per quasi
l'intero dominio osservato, in coerenza con la progettazione di HLL++
\cite{Heule_Nunkesser_Hall_2013}.
Per \(F_0=10^7\) la varianza endpoint è nulla per tutte le configurazioni
best-config perché nel dataset usato vale \(n=d\) e l'endpoint è deterministico
tra run.

\subsection{Metriche endpoint aggregate}

La Tabella~\ref{tab:best-stream-summary} riporta le metriche endpoint mediate
sui seed per ciascun valore finale di \(F_0\).

\begin{table}[htbp]
    \centering
    \scriptsize
    \setlength{\tabcolsep}{4pt}
    \renewcommand{\arraystretch}{1.1}
    \begin{tabular}{|l|r|r|r|r|r|}
        \hline
        Algoritmo & \(F_0\) & MRE medio & RMSE medio & Varianza media & \(\hat{F}_0\) medio \\
        \hline
        HyperLogLog++ & 100000 & 0.001237 & 153.661 & 22550.603 & 99990.816 \\
        HyperLogLog++ & 1000000 & 0.001629 & 2026.870 & 3272587.016 & 1000904.100 \\
        HyperLogLog++ & 5000000 & 0.001287 & 8072.642 & 62887388.036 & 5001168.396 \\
        HyperLogLog++ & 10000000 & 0.001326 & 13256.000 & 0.000 & 10013256.000 \\
        \hline
        HyperLogLog & 100000 & 0.002836 & 358.238 & 127585.802 & 100008.844 \\
        HyperLogLog & 1000000 & 0.003195 & 3938.423 & 15118314.694 & 999213.888 \\
        HyperLogLog & 5000000 & 0.002800 & 17416.878 & 278970064.540 & 4994349.692 \\
        HyperLogLog & 10000000 & 0.002113 & 21134.000 & 0.000 & 9978866.000 \\
        \hline
        LogLog & 100000 & 0.005206 & 649.431 & 407725.504 & 100124.060 \\
        LogLog & 1000000 & 0.005390 & 6789.407 & 46105629.058 & 999245.028 \\
        LogLog & 5000000 & 0.004713 & 28669.821 & 699006423.800 & 4988366.864 \\
        LogLog & 10000000 & 0.007150 & 71500.000 & 0.000 & 9928500.000 \\
        \hline
        Probabilistic Counting & 100000 & 0.579277 & 79492.780 & 6139769255.600 & 117344.080 \\
        Probabilistic Counting & 1000000 & 0.911951 & 1321697.966 & 1418662093360.000 & 1646384.880 \\
        Probabilistic Counting & 5000000 & 0.524089 & 3231428.510 & 10582138707400.000 & 5210955.228 \\
        Probabilistic Counting & 10000000 & 0.084486 & 844860.000 & 0.000 & 10844860.000 \\
        \hline
    \end{tabular}
    \caption{Best-config: metriche endpoint aggregate in modalità streaming (media sui seed).}
    \label{tab:best-stream-summary}
\end{table}

La Figura~\ref{fig:best-stream-mre-by-f0} visualizza lo stesso confronto sul
solo MRE endpoint.

\begin{figure}[htbp]
    \centering
    \includegraphics[width=0.90\textwidth]{figures/results/best_stream_final_mre_by_algorithm_and_f0.png}
    \caption{Best-config: MRE endpoint per algoritmo e valore finale di \(F_0\) (media sui seed).}
    \label{fig:best-stream-mre-by-f0}
\end{figure}

\subsection{Grafici aggiuntivi di revisione scientifica}

La Figura~\ref{fig:best-stream-calibration} mostra la calibrazione endpoint,
ossia \(\hat{F}_0\) medio contro \(F_0\) reale su scala log-log con diagonale
ideale \(y=x\).

\begin{figure}[htbp]
    \centering
    \includegraphics[width=0.78\textwidth]{figures/results/best_stream_endpoint_calibration_loglog.png}
    \caption{Best-config: calibrazione endpoint \(\hat{F}_0\) vs \(F_0\) (media sui seed, scala log-log).}
    \label{fig:best-stream-calibration}
\end{figure}

La Figura~\ref{fig:best-stream-convergence} riporta il primo checkpoint in cui
MRE\((t) \le 5\%\).
HLL e HLL++ soddisfano la soglia già al primo checkpoint disponibile;
LogLog converge più tardi, mentre Probabilistic Counting non raggiunge la soglia
nel dominio analizzato.
Formalmente, il tempo di convergenza è definito come
\[
t_{5\%} = \min\{t \mid \mathrm{MRE}(t)\le 0.05\}.
\]
Nel caso di HLL e HLL++, il valore \(t_{5\%}=1\) va interpretato con cautela:
la soglia è raggiunta subito perché il prefisso iniziale contiene pochissimi
distinti e l'errore relativo medio è ancora poco informativo sul comportamento
asintotico.

\begin{figure}[htbp]
    \centering
    \includegraphics[width=0.82\textwidth]{figures/results/best_stream_convergence_t5pct_loglog.png}
    \caption{Best-config: tempo medio di convergenza alla soglia MRE \(\le 5\%\) (scala log-log).}
    \label{fig:best-stream-convergence}
\end{figure}

\section{Risultati merge con best-config}

Per ogni algoritmo si confrontano stima da merge e stima seriale su coppie di
partizioni.
Le Figure~\ref{fig:best-merge-box} e \ref{fig:best-merge-nonzero} e la
Tabella~\ref{tab:best-merge-summary} mostrano che nel dominio sperimentale
analizzato i delta sono nulli in tutti i casi.
Il risultato è coerente con gli operatori di merge implementati
(massimo componente-per-componente per LogLog/HLL/HLL++ e OR bit-a-bit per
Probabilistic Counting), discussi in
Capitolo~\ref{chp:implementazione} e nel background sulla mergeabilità.

\begin{figure}[htbp]
    \centering
    \includegraphics[width=0.76\textwidth]{figures/results/best_merge_delta_abs_by_algorithm.png}
    \caption{Best-config: distribuzione di \codepath{delta\_merge\_serial\_abs}.}
    \label{fig:best-merge-box}
\end{figure}

\begin{figure}[htbp]
    \centering
    \includegraphics[width=0.76\textwidth]{figures/results/best_merge_nonzero_delta_counts.png}
    \caption{Best-config: conteggio dei casi con delta non nullo nel confronto merge vs seriale.}
    \label{fig:best-merge-nonzero}
\end{figure}

\begin{table}[htbp]
    \centering
    \small
    \setlength{\tabcolsep}{6pt}
    \renewcommand{\arraystretch}{1.12}
    \begin{tabular}{|l|r|r|r|r|r|}
        \hline
        Algoritmo & Righe & Media \(\Delta_{abs}\) & Max \(\Delta_{abs}\) & Media \(\Delta_{rel}\) & Max \(\Delta_{rel}\) \\
        \hline
        HyperLogLog & 500 & 0 & 0 & 0 & 0 \\
        \hline
        HyperLogLog++ & 500 & 0 & 0 & 0 & 0 \\
        \hline
        LogLog & 500 & 0 & 0 & 0 & 0 \\
        \hline
        Probabilistic Counting & 500 & 0 & 0 & 0 & 0 \\
        \hline
    \end{tabular}
    \caption{Best-config: confronto merge vs seriale. Tutti i delta risultano nulli nel dataset analizzato.}
    \label{tab:best-merge-summary}
\end{table}

\section{Discussione}

Nel dominio sperimentale considerato emergono tre risultati principali.
Primo, HyperLogLog++ risulta il metodo più accurato e stabile tra le
configurazioni migliori, con errore medio endpoint inferiore a HyperLogLog e
LogLog su tutti i livelli di \(F_0\) osservati.
Secondo, LogLog resta competitivo ma con una dispersione più alta e una
sensibilità maggiore nelle cardinalità più piccole.
Terzo, Probabilistic Counting rimane sensibilmente più instabile, anche quando
si seleziona il miglior parametro nel dominio testato.

Per la modalità merge, i risultati sono coerenti con la proprietà di fusione
attesa per gli sketch implementati: nel confronto a coppie non si osservano
scarti rispetto al processamento seriale.

\section{Limiti sperimentali}

I limiti principali della campagna corrente sono:
\begin{itemize}
    \item un solo valore di \(n\) nella fase finale (\(10^7\));
    \item un unico profilo dati (distribuzione uniforme, ordine shuffled);
    \item assenza in questo capitolo di misure esplicite di costo temporale e
    footprint di memoria a runtime.
\end{itemize}

Un'estensione naturale è replicare la stessa analisi su distribuzioni non
uniformi e su workload con overlap controllato tra partizioni, mantenendo il
medesimo protocollo di confronto tra endpoint, transitorio e merge.
