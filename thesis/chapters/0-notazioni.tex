%!TEX root = ../dissertation.tex

\chapter*{Notazioni e concetti preliminari}
\addcontentsline{toc}{chapter}{Notazioni e concetti preliminari}
\label{chp:notazioni}

In questa sezione si raccolgono le notazioni utilizzate nel resto della tesi.
\begin{itemize}
    \item $\mathcal{U}$: universo degli elementi.
    \item $S = \langle x_1, \dots, x_s \rangle$: stream di dati.
    \item $S_1 \cdot S_2$: concatenazione di due stream.
    \item $n = |\mathcal{U}|$: dimensione dell'universo.
    \item $f(a)$: frequenza di $a \in \mathcal{U}$.
    \item $(a_t,\Delta_t)$: aggiornamento al tempo $t$, con chiave $a_t$ e incremento $\Delta_t$.
    \item $A_t[j]$: frequenza dell'elemento $j$ dopo i primi $t$ aggiornamenti.
    \item $A_0[j]=0$: condizione iniziale del modello insertion-only.
    \item $f \in \mathbb{N}^{|\mathcal{U}|}$: vettore delle frequenze (componenti $f(a)$).
    \item $\|f\|_1 = \sum_{a\in\mathcal{U}} f(a)$: lunghezza totale della stream.
    \item $F_k$: frequency moments.
    \item $F_0$: numero di distinti nella stream.
    \item $\hat{F}_0$: stima di $F_0$ prodotta da un algoritmo.
    \item $\bar{F}_0$: media dei valori veri su $R$ run.
    \item $\bar{\hat{F}}_0$: media delle stime su $R$ run.
    \item $(\varepsilon,\delta)$: parametri di accuratezza; $\varepsilon$ è l'errore relativo ammesso e $1-\delta$ è la probabilità di successo.
    \item $m$: memoria dello sketch (tipicamente espressa in bit); nelle sezioni su LogLog/HLL/HLL++ il numero di registri è indicato anch'esso con $m=2^p$, quindi lo spazio complessivo dipende anche dalla larghezza di ciascun registro.
    \item $M$: stato interno dell'algoritmo di streaming (lo sketch).
    \item $\mathcal{K}(\cdot)$: procedura che costruisce uno sketch da una stream.
    \item $V$: dominio di uscita di una funzione di hash.
    \item $h:\mathcal{U}\to V$: funzione di hash.
    \item $\mathcal{H}$: famiglia di funzioni di hash.
    \item $L$: lunghezza in bit del valore hash (nel Capitolo 3 si assume $L=w$).
    \item $w$: numero di bit del valore hash (se $V=\{0,1\}^w$).
    \item $p$: parametro di precisione; tipicamente $m=2^p$.
    \item $p'$: precisione usata nella rappresentazione sparsa di HLL++.
    \item $k_{\text{sp}}$: numero di entry non nulle nella rappresentazione sparsa di HLL++.
    \item $\rho(\cdot)$: posizione del primo bit a 1 nel suffisso hash.
    \item $j(x)$: indice del registro selezionato dai $\log_2 m$ bit più significativi di $h(x)$.
    \item $w(x)$: suffisso di $h(x)$ usato per calcolare $\rho(w(x))$ nel registro $j(x)$.
    \item $\alpha_m$: costante di normalizzazione dipendente da $m$.
    \item $\phi$: costante di calibrazione di PCSA ($\phi\approx 0.77351$).
    \item $V_0$: numero di registri a zero (in HLL/HLL++).
    \item $w_{\text{cm}}$: numero di colonne nel Count-Min Sketch.
    \item $d$: numero di righe/funzioni hash nel Count-Min Sketch.
    \item $m_{\text{bf}}$: numero di bit del Bloom filter.
    \item $k_{\text{bf}}$: numero di funzioni hash del Bloom filter.
    \item $n_{\text{ins}}$: numero di elementi inseriti in un Bloom filter.
    \item $\mathcal{S}$: spazio degli stati di uno sketch.
    \item $\oplus$: operatore di merge tra stati di sketch.
    \item $R$: numero di run (ripetizioni) sperimentali.
    \item $\sigma$: deviazione standard campionaria delle stime.
    \item $\widehat{\mathrm{Var}}(\hat{F}_0)$: varianza campionaria delle stime su $R$ run.
    \item $\widehat{\sigma}$: deviazione standard campionaria (\(\widehat{\sigma}=\sqrt{\widehat{\mathrm{Var}}(\hat{F}_0)}\)).
    \item $\mathrm{RE}$: errore relativo.
    \item $\mathrm{RSE}$: relative standard error.
    \item $\mathrm{RSE}_{\text{obs}}$: relative standard error osservata, definita come \(\widehat{\sigma}/\bar{F}_0\).
    \item $\mathrm{Bias}(\hat{\theta})$: bias di uno stimatore.
    \item $\mathrm{Var}(\hat{\theta})$: varianza di uno stimatore.
    \item $\widehat{\mathrm{Bias}}(\hat{F}_0)$: stima empirica del bias su $R$ run.
    \item $\mathrm{AB}$: absolute bias.
    \item $\mathrm{RB}$: relative bias.
    \item $\mathrm{MRE}$: mean relative error.
    \item $\mathrm{MAE}$: mean absolute error.
    \item $\mathrm{RMSE}$: root mean squared error.
\end{itemize}
